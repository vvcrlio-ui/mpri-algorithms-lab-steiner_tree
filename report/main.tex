\documentclass{article}

\usepackage[english]{babel}
\usepackage{ctex}      % 中文支持
\usepackage{amsmath}   % 数学公式
\usepackage{amssymb}   % 数学符号,例如 \mathbb
\usepackage{titling}   % 自定义标题

\pretitle{\begin{center}\LARGE\bfseries}
\posttitle{\end{center}}

\title{MPRI Algorithms Lab: The Steiner Tree Problem}
\author{Xiang Wan} 

\begin{document}

\maketitle % 这行命令会根据上面的标题和作者信息生成一个标题页

\section{Problem Modeling}

\subsection{Problem Definition: Minimum Steiner Tree}
\label{sec:problem_definition_en}

The following is the standard formal definition of the Minimum Steiner Tree problem, as found in the compendium.

\begin{description}
    \item[Instance:] 
    A complete graph $G=(V,E)$, a metric given by edge weights $w: E \to \mathbb{Z}^+$, and a subset of required vertices $S \subset V$.

    \item[Solution:] 
    A Steiner tree, i.e., a subtree of $G$ that includes all the vertices in $S$.

    \item[Measure:] 
    The sum of the weights of the edges in the subtree, which is to be minimized.

    \item[Good News:] 
    The problem is approximable within a ratio of $1+\ln(3)/2 \approx 1.55$. In your project file, this approximation ratio is simplified to 2.

    \item[Bad News:] 
    The problem is APX-complete.

    
    \item[Garey and Johnson:] ND12.

\end{description}


% --- 正文结束 ---
\end{document}