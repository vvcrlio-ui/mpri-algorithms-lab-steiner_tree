\documentclass{article}

\usepackage[english]{babel}
\usepackage{ctex}      % 中文支持
\usepackage{amsmath}   % 数学公式
\usepackage{amssymb}   % 数学符号,例如 \mathbb
\usepackage{titling}   % 自定义标题

\pretitle{\begin{center}\LARGE\bfseries}
\posttitle{\end{center}}

\title{MPRI Algorithms Lab: The Steiner Tree Problem}
\author{Xiang Wan}
\date{\today} % 添加了自动日期

\begin{document}

\maketitle % 这行命令会根据上面的标题和作者信息生成一个标题页

\section{Problem Modeling}

\subsection{Problem Definition: Minimum Steiner Tree}
\label{sec:problem_definition_en}

The following is the standard formal definition of the Minimum Steiner Tree problem, as found in the compendium.

\begin{description}
    \item[Instance:] 
    A complete graph $G=(V,E)$, a metric given by edge weights $w: E \to \mathbb{Z}^+$, and a subset of required vertices $S \subset V$.

    \item[Solution:] 
    A Steiner tree, i.e., a subtree of $G$ that includes all the vertices in $S$.

    \item[Measure:] 
    The sum of the weights of the edges in the subtree, which is to be minimized.

    \item[Good News:] 
    The problem is approximable within a ratio of $1+\ln(3)/2 \approx 1.55$. In your project file, this approximation ratio is simplified to 2.

    \item[Bad News:] 
    The problem is APX-complete.
    
    \item[Garey and Johnson:] ND12.
\end{description}

\subsection{NP-completeness Proof}
\label{sec:np_completeness_proof}

To prove that the Minimum Steiner Tree problem (as stated above: \emph{complete graph} with \emph{metric} integer weights) is NP-complete we do two things:
\begin{enumerate}
    \item Show membership in NP.
    \item Provide a polynomial-time reduction from the NP-complete \textbf{Exact Cover} problem to our problem statement. The reduction is carried out in two steps: (i) build a general graph with positive integer weights (no zero weights) via a standard construction from Exact Cover; (ii) take the \emph{metric closure} (shortest-path completion) of that graph to obtain a complete graph whose edge weights are positive integers and satisfy the triangle inequality. We prove these transformations preserve the existence of a Steiner tree within the chosen budget.
\end{enumerate}

\subsubsection{Membership in NP}
Given an instance $(G=(V,E),w,R,B)$ of the Steiner Tree problem (here $G$ is a complete metric graph and $w:E\to\mathbb Z^+$), a certificate is a subtree $T$ of $G$ (for example given as a list of edges). We can verify in polynomial time:
\begin{itemize}
    \item that the listed edges form a connected acyclic subgraph (a tree) on the claimed vertex set,
    \item that every terminal in $R$ appears in the vertex set of $T$,
    \item that the sum of the weights of the edges of $T$ is at most $B$.
\end{itemize}
Thus the problem is in NP.

\subsubsection{Reduction from Exact Cover (two-step: graph construction + metric closure)}
We start from an instance of Exact Cover.

\paragraph{Exact Cover (decision) instance.}
Let the instance be a universe $U=\{u_1,\dots,u_n\}$ and a collection of subsets $\mathcal S=\{S_1,\dots,S_k\}$. The question is whether there exists a subcollection $\mathcal S'\subseteq\mathcal S$ such that the sets in $\mathcal S'$ are pairwise disjoint and their union equals $U$.

\paragraph{Step 1: Construct a (sparse) weighted graph $G'=(V',E')$ with positive integer weights.}
We construct $G'$ as follows (this construction is polynomial in the size of the Exact Cover instance):
\begin{itemize}
    \item Vertices:
    \[
      V'=\{a_0\}\ \cup\ \{s_1,\dots,s_k\}\ \cup\ \{u_1,\dots,u_n\}.
    \]
    (Here $a_0$ is a special root vertex, each $s_i$ corresponds to subset $S_i$, and each $u_j$ corresponds to element $u_j\in U$.)
    \item Edges and weights (all weights are positive integers):
    \begin{itemize}
        \item For each $i\in\{1,\dots,k\}$ add edge $(a_0,s_i)$ with weight
        \[
          w'(a_0,s_i)=|S_i|.
        \]
        \item For each $i$ and each $u_j\in S_i$ add edge $(s_i,u_j)$ with weight
        \[
          w'(s_i,u_j)=1.
        \]
        \item No other edges are added in $G'$.
    \end{itemize}
    \item Terminals and budget:
    \[
      R=\{a_0,u_1,\dots,u_n\},\qquad B=2n.
    \]
\end{itemize}
All weights in $G'$ are positive integers and the graph size is polynomial in $n+k$.

\paragraph{Correctness of Step 1 (equivalence between Exact Cover and Steiner tree in $G'$ with budget $B=2n$).}
We show:
\[
  \text{Exact cover exists} \quad\iff\quad
  \text{there exists a Steiner tree in } G' \text{ that connects } R \text{ with total weight }\le 2n.
\]

\textbf{($\Rightarrow$)} Suppose there is an exact cover $\mathcal S'\subseteq\mathcal S$. Then the sets in $\mathcal S'$ are disjoint and their union is $U$, so $\sum_{S_i\in\mathcal S'}|S_i|=n$. Construct a subtree $T'$ of $G'$ as follows: include edges $(a_0,s_i)$ for each $S_i\in\mathcal S'$, and for each element $u_j\in U$ include the unique edge $(s_i,u_j)$ where $S_i\in\mathcal S'$ is the unique set containing $u_j$. The resulting subgraph connects $a_0$ to every $u_j$ and is acyclic (it is a forest that can be pruned to a tree if necessary); its total weight equals
\[
  \sum_{S_i\in\mathcal S'} w'(a_0,s_i) \;+\; \sum_{j=1}^n w'(s_{i(j)},u_j)
  \;=\;\sum_{S_i\in\mathcal S'} |S_i| \;+\; n \;=\; n + n \;=\; 2n.
\]
Thus there is a Steiner tree of weight $2n$ (hence $\le 2n$).

\textbf{($\Leftarrow$)} Conversely, suppose there exists a Steiner tree $T'$ in $G'$ that connects all terminals $R$ and has total weight $W(T')\le 2n$. Observe:
\begin{itemize}
    \item Every terminal $u_j$ has neighbors only among the vertices $\{s_i : u_j\in S_i\}$ in $G'$. Therefore in any connected subgraph that contains $a_0$ and $u_j$, the path from $a_0$ to $u_j$ must use at least one edge of the form $(s_i,u_j)$ (for some $i$ with $u_j\in S_i$). Hence the tree $T'$ contains at least one $(s_i,u_j)$ edge for each element $u_j$, so the total contribution to $W(T')$ from edges of type $(s_i,u_j)$ is at least $n$ (each has weight $\ge 1$ and there are at least $n$ of them).
    \item Let $\mathcal S''$ be the collection of sets corresponding to those $s_i$ for which $T'$ contains the edge $(a_0,s_i)$; these are exactly the subset-vertices used to connect $a_0$ into the rest of the tree. Because every $u_j$ must be connected to $a_0$ in $T'$, every $u_j$ must belong to at least one set in $\mathcal S''$, so $\mathcal S''$ covers $U$. Therefore the contribution to $W(T')$ from edges of type $(a_0,s_i)$ is
    \[
      \sum_{S_i\in\mathcal S''} w'(a_0,s_i)
      \;=\;\sum_{S_i\in\mathcal S''} |S_i|
      \;\ge\; n.
    \]
\end{itemize}
Combining the two contributions yields \(W(T') \ge n + n = 2n\). Since we assumed \(W(T')\le 2n\), we must have equality throughout. Equality implies (i) the number of $(s_i,u_j)$ edges used is exactly $n$ (so for each $u_j$ exactly one such edge is used in $T'$), and (ii) \(\sum_{S_i\in\mathcal S''}|S_i|=n\). The latter together with the fact that $\mathcal S''$ covers $U$ forces the sets in $\mathcal S''$ to be pairwise disjoint and to partition $U$; hence $\mathcal S''$ is an exact cover. This proves the equivalence.

Thus Exact Cover reduces in polynomial time to the Steiner Tree problem on the general graph $G'$ with positive integer weights and budget $2n$.

\paragraph{Step 2: From the general graph $G'$ to a complete metric graph $G^*$ (metric closure).}
We now transform $G'$ into a complete graph $G^*=(V',E^*)$ on the same vertex set $V'$ by assigning to every unordered pair $\{x,y\}$ the weight equal to the shortest-path distance in $G'$ (with respect to $w'$):
\[
  w^*(x,y):=\operatorname{dist}_{G'}(x,y)=\min\{\text{sum of }w'\text{-weights along a path from }x\text{ to }y\}.
\]
Since all $w'$ are positive integers, all distances $w^*(x,y)$ are positive integers. By construction $w^*$ satisfies the triangle inequality, so $(V',w^*)$ is a metric and $G^*$ is a complete metric graph matching the format required by the problem statement in Section~\ref{sec:problem_definition_en}.

We must show the existence of a Steiner tree for terminals $R$ of weight at most $2n$ in $G'$ is equivalent to the existence of a Steiner tree for $R$ of weight at most $2n$ in $G^*$. The standard argument is:

\begin{itemize}
  \item If $T$ is any Steiner tree in $G'$, then viewing $T$ as a subgraph of $G^*$ (each edge of $T$ also corresponds to a pair of vertices in $G^*$) we have $w^*(x,y)\le w'(x,y)$ for every edge $(x,y)\in T$. Hence the total weight of the same edge set measured under $w^*$ is at most its weight under $w'$. Therefore a feasible Steiner tree in $G'$ with weight $\le 2n$ yields a feasible Steiner tree in $G^*$ with weight $\le 2n$.
  \item Conversely, if $T^*$ is a Steiner tree in $G^*$ with weight $W^*\le 2n$, replace each edge $(x,y)$ of $T^*$ by a shortest path between $x$ and $y$ in $G'$. The union $H$ of these shortest paths is a connected subgraph of $G'$ whose total $w'$-weight equals $W^*$. From $H$ we can extract a spanning tree (on the vertices involved) by removing cycles; removing edges cannot increase total weight. The result is a tree in $G'$ connecting all terminals $R$ with total weight $\le W^* \le 2n$. Hence a feasible solution in $G^*$ corresponds to a feasible solution in $G'$.
\end{itemize}

Therefore the two-step reduction (Exact Cover $\to G'$ with positive integer weights $\to$ metric closure $G^*$) produces in polynomial time a complete metric graph instance $(G^*,w^*,R,B=2n)$ that has a Steiner tree of weight at most $2n$ iff the original Exact Cover instance has a solution.

\subsubsection{Conclusion}
Exact Cover is NP-complete; we have given a polynomial-time reduction from Exact Cover to the Minimum Steiner Tree problem as stated in Section~\ref{sec:problem_definition_en} (complete graph, metric weights in $\mathbb Z^+$). Combined with the observation that Steiner Tree belongs to NP, it follows that the Minimum Steiner Tree problem (in the formulation used in this document) is NP-complete.

% --- 参考文献 ---
\begin{thebibliography}{9}

\bibitem{Karp1972}
Richard M. Karp.
\newblock Reducibility Among Combinatorial Problems.
\newblock In M. J{\"u}nger et al. (eds.), \emph{50 Years of Integer Programming 1958-2008}, pages 219--241. Springer-Verlag Berlin Heidelberg, 2010.
\newblock (Reprint of the original 1972 paper).

\end{thebibliography}

\end{document}
